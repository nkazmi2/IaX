\documentclass[preprint2]{aastex}

\newcommand{\vdag}{(v)^\dagger}
\newcommand{\myemail}{nkazmi2@illinois.com}
%{***YOUR EMAIL HERE***}

\shorttitle{Abstract Challenge}
%{***PUT AN SHORTENED TITLE HERE***}
\shortauthors{***FIRST AUTHOR*** et al.}

\begin{document}

\title{Challenge: Supernovae Iax }
%{***PUT TITLE HERE***}

\author{Kazmi, N. F.} 
\affil{Dept. of Astronomy, University of Illinois at Urbana Champaign, Urbana, IL 61801}
%\author{***FIRST AUTHOR'S LAST NAME, FIRST INITIAL MIDDLE INITIAL(?)***}
%\affil{***FIRST AUTHOR'S DEPARTMENT, INSTITUTION, CITY, STATE CODE ZIP 
%CODE***}
%\author{***SECOND AUTHOR'S LAST NAME, FIRST INITIAL MIDDLE 
%INITIAL(?)***}
%\affil{***SECON AUTHOR'S DEPARTMENT, INSTITUTION, CITY, STATE CODE ZIP 
%CODE***}
%\author{***THIRD AUTHOR'S LAST NAME, FIRST INITIAL MIDDLE INITIAL(?)***}
%\affil{***THIRD AUTHOR'S DEPARTMENT, INSTITUTION, CITY, STATE CODE ZIP 
%CODE***}
%repeat as necessary

\begin{abstract}
\noindent We present an analysis of Supernovae (SNe) 2008ge, 2008ha, 2010ae, 
2010el. These SNe were observed using the Hubble Space Telescope (HST) in the
%f435w, f555w, f625w, and f814w wavebands.  
optical wavebands.
These SNe are catagorized as Type Iax, a distinct category from Type Ia or Type II. 
Iax are notible for their low spectra luminosity and low ejecta velocities. 
The stellar populations around these four SNe were analyzed to understand
 the ages of the SNe. 
There is a correlation between the ages of the four separate SNe. 
The details of the analysis include an examination of wavebands ; 
adjusting the spectra for host and Milky Way (MW) reddening; 
and filtering out sources with respect to their crowding, sharpness, and
 roundness features.
%The occurance and trend in ages shows that IaX are a common feature
% for all SNe.
 Further study needs to be done to have conclusive results. \\

%\smallskip
\noindent\textit{Key words}: supernovae: general - supernovae: individual (SN 2008ge, SN 2008ha, 
SN 2010ae, SN 2010el)
\end{abstract}.

\section{Introduction}
Thermonuclear explosions, Type Ia Supernovae (SNe Ia), have been studied more in
 the last decade due to the increase in technology and an interest in the accelerating
 universe.
Type Ia are used to measure the rate of expansion in the universe
(Perlmutter et al. 1999) because of their near uniform light curves. 
However, not all thermonuclear explosions share this behaviour. 
%Continued study of thermonuclear explosions reveal that they are not a "simple"
% catagory of SNe, they show a wide range of behaviours and spectral features. 
Continued study of thermonuclear explosions reveal a range of behaviours 
and spectral features. 

The type of SNe that will be discussed in this paper is SNe Type Iax, 
these are a distinct catagory from Type Ia (Thermonuclear explosions)
and Type II (Core-collapse SNe). 
Iax possesses a less luminous light curve and a slower decline rate (SOURCE).
They have a lower ejecta velocity and do not show hydrogen and helium in their
 spectra unless the elements were brought in by outside factors,
such as circumstellar disks or intermediate mass elements (SOURCE). 
Iax do not have an observable second maxima in the near-infared (NIR), 
(e.g., Li et al. 2003).
The explosion mechanics of Iax are still unknown.
They closest model predicts accretion from a Helium star onto a Carbon-Oxygen
 White Dwarf (C/O WD) (SOURCE). 
The unique features of Iax indicate that they are not a subset of Ia, 
but should instead be considered their own class. 
%The differences between Ia and Iax indicate that 
%However, Iax are predicted to be a unique "set" of thermonuclear explosions. 
%While the mechanics of this model have not been finalized, 
%the spectra and lightcurves of Iax show that it is a distinct type of SNe
%explosion, and not a subset of Ia. 

Earlier literature called Iax, 'SN2002cx-like'. SN2002cx was the first SN to be heavily 
studied because of it's peculiar characteristics. 
%[Now list it's characteristics] 
%SN 2008ha VS SN 2002cx
%[SN 2008ha has an expansion velocity ∼3000 km s−1 lower than that of SN 2002cx,
%which has expansion velocities of ∼5000 km s−1 compared
%to ∼10,000 km s−1 for a normal SN Ia. SN 2008ha has a
%very small peak absolute magnitude (M ≈ −14 mag), while
%SN 2002cx has a peak absolute magnitude of M ≈ −17 mag,
%∼2 mag below that of normal SNe Ia, but extending the
%relationship between decline rate and luminosity (Phillips 1993),]
These distinctions from Ia and the similarities between newly observed SNe 
created the class Iax. 
Four SNe from this class were studied in this analysis, SN2008ge, SN2008ha, 
SN2010ae, and SN2010el. 
The focuse was on contraining the age limit of the
surrounding stellar populations of these four SNe.

The stellar population around SN2008ha (08ha) was modeled prior to this analysis. 
08ha (why was this modeled more? Lets assume you know).
The constraints from 08ha were applied to 08ge, 10ae, and 10el. 

In this paper, we will discuss our findings of Iax stellar populations
% (this is a weird sentence).
 In section 2, we will discuss the Spectroscopy and Photometry of these SNe and 
compare it with earlier results. 
We will discuss the methods of our analysis in section 3. 
In section 4, we will discuss our contraints and results.
The summary of this project is in section 5. 
The World Coordinate System (WCS) is used throughout this analysis. 
%Reddening shift of the MilkyWay and host galaxies are used
%to measure the extinction of the spectra.

%WHAT OTHER PAPERS HAVE STUDIED THIS - citations

\section{Class Members}
The classification of SNe was based off of spectral observations.
The increase in SNe data shows the variations in SNe spectra. 
SN Iax are a result of the new classifications, Iax had a much lower
 light curve, (exact value), and other stuff.  
The current list of SNe Type Iax contain 25 SNe,
of that list, 4 SNe were chosen to study in depth. 

\subsection{SN 2008ge}
SN 2008ge was first observed by CHASE on 2008
October 8.27 in NGC 1527 (Pignata et al. 2008).

\subsection{SN 2008ha}

SN 2008ha was first observed 2008 November 7.17 in UGC 12862 
(Puckett et al. 2008).
The classification of Iax was originally called "SN 2002cx-like",
SN 2008ha falls on the extreme end of this type. 
It is less luminous and has one of the smallest ejecta velocity. 
(Foley et al. 2009, 2010a; Valenti et al. 2009). 
The remaining criteria (what makes it Iax).

\subsection{SN 2010ae}
SN 2010ae was first observed 2010 February 23 in ESO 162-G017
(Pignata et al. 2010) [direct line from foley 2013].
First observations of it declared it a SN Ia, later it was identified to have
a spectra similar to SN 2008ha (Stritzinger et al. 2010b). 
It was then reclassified as a SN Iax.

\subsection{SN 2010el}
SN 2010el was first observed 2010 June 19 in NGC 1566
(Monard 2010) [direct line from foley 2013]
It was identified to have a spectra similar to SN 2008ha and was 
catagorized as Type Iax (Bessell et al. 2010).

\section{Spectroscopy \& Photometry}

This analysis used HST photometric images of SNe 2008ge, 2008ha, 
2010ae, and 2010el. 
These four SNe were observered in the F435W, F555W, F625W, 
and F814W wavebands. 
The HST collected these images using (HST Info- exposures, camera lens size 
and date and time it observed each object.
Are these early time or late time spectra, what part of the timeline are we 
looking at)
This data was collected [DATE] using [type of collection method, ccd stuff,
how the images were corrected]. 
From these images we were able to generate a catalog of sources using the
DoPHOT photometry package. 

\section{Methods}
The catalogs generated for this analysis provided us with a selection of stars.
The catalog  was filtered out to focus on sources, bright and faint stars.
The remaining criteria included it's position, the filters included sources within a radius 
100 pixels, though the physical distance is dependent on the SNe being discussed. 
The other criteria was the Signal to Noise (S/N), at minimum, 
sources had to be above S/N of 3 in at least one waveband. 
The sources also needed to have particular Crowding values, for example, 
very clustered sources may included extended sources, which would not be "good" 
sources. The sharpness and roundness was also taken into account. 
The sharpness was set to [], and the roundness was set to []. 

Using these parameters, we were able to recreate the Color Magnitude Diagram (CMD)
of SN 2008ha, figure () from (Foley ...). This study focused on the stars within
 (15, 30, and 45 pix -> turn to pc)


\section{Discussion}
The stellar populations around SNe 2008ge, 2008ha, 2010ae, and 2010el

\section{Results}

{\it Acknowledgments:} We are grateful to **ACKNOWLEDGEMENTS HERE***.

%\begin{thebibliography}{}
%Example
%\bibitem[Franchetti et al.(2012)]{Fetal12} 
%Franchetti, N.~A., Gruendl, R.~A., Chu, Y.-H., et al.,\ 2012, \aj, 143, 85 
%\end{thebibliography}

\end{document}