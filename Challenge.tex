\documentclass[preprint2]{aastex}

\newcommand{\vdag}{(v)^\dagger}
\newcommand{\myemail}{nkazmi2@illinois.com}
%{***YOUR EMAIL HERE***}

\shorttitle{Abstract Challenge}
%{***PUT AN SHORTENED TITLE HERE***}
\shortauthors{***FIRST AUTHOR*** et al.}

\begin{document}

\title{Abstract Challenge: Supernovae Iax }
%{***PUT TITLE HERE***}

\author{Kazmi, N. F.}
\affil{Dept. of Astronomy, University of Illinois at Urbana Champaign, Urbana, IL
 61801}
\author{Foley, R}
\affil{Dept. of Astronomy, University of Illinois at Urbana Champaign, Urbana, IL 
61801}
%\author{***FIRST AUTHOR'S LAST NAME, FIRST INITIAL MIDDLE INITIAL(?)***}
%\affil{***FIRST AUTHOR'S DEPARTMENT, INSTITUTION, CITY, STATE CODE ZIP 
%CODE***}
%\author{***SECOND AUTHOR'S LAST NAME, FIRST INITIAL MIDDLE 
%INITIAL(?)***}
%\affil{***SECON AUTHOR'S DEPARTMENT, INSTITUTION, CITY, STATE CODE ZIP 
%CODE***}
%\author{***THIRD AUTHOR'S LAST NAME, FIRST INITIAL MIDDLE INITIAL(?)***}
%\affil{***THIRD AUTHOR'S DEPARTMENT, INSTITUTION, CITY, STATE CODE ZIP 
%CODE***}
%repeat as necessary

\begin{abstract}
%*WRITE ABSTRACT HERE*
*First thoughts, not even close to a first draft*\\
We present analysis of Hubble Space Telescope (HST) images of Supernovae (SNe)
 2010el, 2010ae, 2008ha, and 2008ge. These SNe are catagorized as Type Iax, a
 distinct category from Ia or II. Iax are notible for their low spectra luminosity and low
 ejecta velocities. The stellar populations around these four SNe were analyzed to
 understand the ages of the SNe. There is a correlation between the ages of the four
 separate SNe. The details of the analysis include an examination of wavebands 
 f435w, f555w, f625w, and f814w; adjusting the spectra for host and Milkyway (MW)
 reddening; and filtering out sources with respect to their crowding, sharpness, and
 roundness. The occurance and trend in ages shows that IaX are a common feature for
 all SNe. Further study needs to be done to have conclusive results. 

\end{abstract}.


\keywords{***LOOK UP HOW TO USE KEYWORDS***}

\section{Introduction}
BACKGROUND OF WHAT IAX IS
originally called ‘SN2002cx-like'
revolutionized cosmology with the discovery that the
expansion of the universe is currently accelerating, driven by an unknown energy
(Riess et al. 1998; Perlmutter et al. 1999).
 no understanding of SN Ia progenitor systems and explosion mechanism


WITH RESPECT TO TYPE IA
WHAT ARE THEIR SIMILARITIES
WHAT ARE THEIR DIFFERENCES
diverse population of hydrogen
deficient, rapidly evolving supernovae
WHAT OTHER PAPERS HAVE STUDIED THIS - citations
Supernovae Type Iax (SNe Iax) are thermonuclear explosions, however they have
 unique qualities that separate them from traditional thermonuclear supernovae,
 Supernovae Ia (SNe Ia). SNe Iax are less luminous and have a lower ejecta velocity
than their Ia counterparts. Iax do not show hydrogen and helium in their spectra,
 unless the elements were brought in by outside factors, such as circumstellar disks or
 intermediate mass elements. Iax have been identified more in the last decade and are
estimated to occur 30$\%$ out of the total number of thermonuclear explosions 
[or all explosions?] . In this paper, we will focus on contraining the age limit of the
surrounding stellar populations

 The ability to collapse the observa
tional diversity of this class to one or two parameters suggests
that most SNe Ia have similar progenitor stars (although not
necessarily progenitor systems, as some SN Ia observables cor
relate with their progenitor environment;  Foley et al. 2012b)
and explosion mechanisms. 

no observed second maximum in
the near-infrared (NIR) bands (e.g., Li et al. 2003); late-time
spectra dominated by narrow permitted Fe ii lines (Jha et al.
2006; Sahu et al. 2008), but can occasionally have strong [Fe ii]
emission (Foley et al. 2010c); strong mixing of the ejecta (Jha
et al. 2006; Phillips et al. 2007); and a host-galaxy morphology
distribution highly skewed to late-type galaxies, and no member
of this class has been discovered in an elliptical galaxy (Foley
et al. 2009; Valenti et al. 2009). Additionally, some members of
the class, such as SN 2007J, display strong He i lines in their
spectra (Foley et al. 2009).

SNe Iax are not simply a subclass of SNe Ia
Photometrically, normal SNe Ia show a correlation between
the peak luminosity and light-curve decline rate.

 low kinetic energy and significant mixing
in the ejecta, may be consistent with a full deflagration of a white
dwarf (WD; Branch et al. 2004; Phillips et al. 2007),



, rather than
a deflagration that transitions into a detonation as expected for
normal SNe Ia (Khokhlov 1991). Because of their low velocities,
which eases line identification and helps probe the deflagration
process, which is essential to all SN Ia explosions, this class is
particularly useful for understanding typical SN Ia explosions.

We present and discuss our [UVW1]UBVRIJHK photometry
in Section and optical spectroscopy in Section 3. In Section 4,
we discuss the energetics of SN 2008ha. We examine the host
galaxy of SN 2008ha and that of all SN 2002cx-like objects in
Section 5 , in sec 6, 
\\

The manuscript is structured in the following way. Section 2
outlines the criteria for membership in the class and details
the members of the class. We present previously published
and new observations of the SNe in Section 3. We describe
the photometric and spectroscopic properties of the class in
Sections 4 and 5, respectively. In Section 6, we provide estimates
of the relative rate of SNe Iax to normal SNe Ia and the Fe
production from SNe Iax. We summarize the observations and
constrain possible progenitor systems in Section 7, and we
conclude in Section 8. UT dates are used throughout the paper


The final results of this project are concluded in section [num]. 
08ha stuff
SN 2008ha is similar in many ways to
SN 2002cx, but there are some differences. SN 2008ha has an
expansion velocity ∼3000 km s−1 lower than that of SN 2002cx,
which has expansion velocities of ∼5000 km s−1 compared
to ∼10,000 km s−1 for a normal SN Ia. SN 2008ha has a
very small peak absolute magnitude (M ≈ −14 mag), while
SN 2002cx has a peak absolute magnitude of M ≈ −17 mag,
∼2 mag below that of normal SNe Ia, but extending the
relationship between decline rate and luminosity (Phillips 1993),

Throughout this paper we adopt H0 = 73 km s−1 Mpc−1 and
correct redshifts (z) using the Virgo+GA infall model of Mould
et al. (2000) via NED32 to estimate distances to the SN host
galaxies.

We recognized the uniqueness of SN 2002cx (SN 1991T–like
premaximum spectrum, SN 1991bg–like luminosity, and very
low expansion velocity) shortly after its discovery, and a fol
low-up program of multicolor photometry was established at
Lick Observatory. Spectra of the SN were obtained with the
FLWO 1.5 m telescope and also with the Keck 10 m telescopes.
This paper presents the results from this campaign and is or
ganized as follows. Section 2 contains a description of the
observations and analysis of the photometry, including our
methods of performing photometry, our calibration of the mea
surements onto the standard Johnson-Cousins system, our re
sulting multicolor light curves, and our comparisons between
the light curves and color curves of SN 2002cx and those of
other SNe Ia. Section 3 contains a description of the spectral
observations and analysis. We discuss the implications of our
observations in § 4 and summarize our conclusions in § 5.


It is clear from the above discussion that the association with
different types of stellar environment is of key importance in
distinguishing between these different types of luminous tran
sients and in constraining the possible progenitor objects. However,
much of the environmental information, e.g. the association of the
Ca-rich transients with old populations and SN2002cx-like tran
sients with young, lacks quantification and in many cases is little
more than anecdotal. Host galaxy classifications give some useful
information, but they are notoriously subjective and, even if free
from actual errors, they do not give precise information on the stel
lar population at the location of the transient event. For example,
even a late-type spiral may have a bulge, or extreme outer disc, that
is entirely composed of old stars. In this paper, we will make use
of both host galaxy types and quantified measures of star-formation
(SF) activity, local to the sites of events within their host galaxies,
applied specifically to the known samples of Ca-rich and SN2002cx
like transients, to determine whether they appear to rise from the
same progenitor populations and to compare these populations with
the same measures for other types of SN (including ‘normal’ SNeIa,
and core-collapse-type SNe Ib, Ic and II-P).
















\section{Methodology}

***METHODOLOGY TEXT***

\section{Observations and Data Reduction}

This analysis used Hubble Space Telescope (HST) images in the F435W, F555W, F625W,
 and F814W wavebands. This data was collected [DATE] using [type of collection
 method, ccd stuff]. \\
From these images we were able to generate a catalog of sources in the 

\section{Discussion}

***METHODOLOGY TEXT***
{\it Acknowledgments:} We are grateful to **ACKNOWLEDGEMENTS HERE***.

%\begin{thebibliography}{}
%Example
%\bibitem[Franchetti et al.(2012)]{Fetal12} 
%Franchetti, N.~A., Gruendl, R.~A., Chu, Y.-H., et al.,\ 2012, \aj, 143, 85 
%\end{thebibliography}

\end{document}