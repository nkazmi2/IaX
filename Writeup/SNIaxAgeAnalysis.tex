\documentclass[preprint]{aastex}
\usepackage{amsmath,natbib}
%\usepackage{cite}
\usepackage{graphicx}
\usepackage[caption=false]{subfig}

%\setcitestyle{authoryear}
%\usepackage{cite}
%\documentstyle[natbib209]
%\citestyle{aa}
%\newcommand{\vdag}{(v)^\dagger}
%\newcommand{\myemail}{nkazmi2@illinois.com}
%{***YOUR EMAIL HERE***}

\shorttitle{Supernovae Iax Age Analysis}
%{***PUT AN SHORTENED TITLE HERE***}
\shortauthors{Kazmi, N. F. et al.}

\begin{document}

\title{Supernovae Iax Age Analysis}
%{***PUT TITLE HERE***}

\author{Novarah F. Kazmi\altaffilmark{1}, Ryan J. Foley\altaffilmark{1}, Curtis McCully\altaffilmark{2}}
\altaffiltext{1}{Astronomy Department, University of Illinois at Urbana-Champaign,
1002 W. Green Street, Urbana, Illinois 61801, USA.}
\altaffiltext{2}{Department of Physics, University of California, Santa Barbara, CA 93106}
%{Department of Physics and Astronomy, Rutgers, the State University
%of New Jersey, 136 Frelinghuysen Road, Piscataway, New Jersey
%08854, USA}
%\author{***FIRST AUTHOR'S LAST NAME, FIRST INITIAL MIDDLE INITIAL(?)***}
%\affil{***FIRST AUTHOR'S DEPARTMENT, INSTITUTION, CITY, STATE CODE ZIP 
%CODE***}
%\author{***SECOND AUTHOR'S LAST NAME, FIRST INITIAL MIDDLE 
%INITIAL(?)***}
%\affil{***SECOND AUTHOR'S DEPARTMENT, INSTITUTION, CITY, STATE CODE ZIP 
%CODE***}
%\author{***THIRD AUTHOR'S LAST NAME, FIRST INITIAL MIDDLE INITIAL(?)***}
%\affil{***THIRD AUTHOR'S DEPARTMENT, INSTITUTION, CITY, STATE CODE ZIP 
%CODE***}
%repeat as necessary

\begin{abstract}
\noindent Rewriting Abstract%We present an analysis of Supernovae (SNe) 2008ge, 2008ha, 2010ae, 
%2010el. These SNe were observed using the Hubble Space Telescope (HST) in the
%%f435w, f555w, f625w, and f814w wavebands.  
%optical wavebands.
%These SNe are catagorized as Type Iax, a distinct category from Type Ia or Type II. 
%Iax are notible for their low spectra luminosity and low ejecta velocities. 
%The stellar populations around these four SNe were analyzed to understand
% the ages of the SNe. 
%There is a correlation between the ages of the four separate SNe. 
%The details of the analysis include an examination of wavebands ; 
%adjusting the spectra for host and Milky Way (MW) reddening; 
%and filtering out sources with respect to their crowding, sharpness, and
% roundness features.
%%The occurance and trend in ages shows that IaX are a common feature
%% for all SNe.
% Further study needs to be done to have conclusive results. \\

%\smallskip
\noindent\textit{Key words}: supernovae: general - supernovae: individual (SN 2008ge, SN 2008ha, 
SN 2010ae, SN 2010el)%, SN 2012Z, SN 2014dt)
\end{abstract}

\section{Introduction}
A relatively new developement in Supernovae (SNe) classification 
has been the Type Iax Supernovae (SNe Iax). 
These are separate from Type Ia SNe (no hydrogen lines, thermonuclear explosions) 
and Type II SNe (promenint hydrogen features, core-collapse), 
%SNe which are classified by their optical spectra then higlighted by the detonation mechanisms.
Iax are theorized to be thermonuclear explosions of Carbon-Oxygen
White Dwarf (C/O WD) and they share spectral properties with Ia SNe; however,
they have unique features which separate them from the traditional Ia SNe \citep{fol1304}. 

The differences between Ia and Iax begin with their luminosities.
Iax SNe possess a less luminous light curve, 
their absolute magnitudes fall between a range of 
``$-14.2 \leq M_{V,peak} \leq -18.4$ mag''
 \citep{fol1304}.
%The Iax rise-time rate is slightly larger than their Ia counterparts, 
%, a slow decline rate,
% NO CLEAR IDEA OF WHAT IS CORRECT FOR RISE/DECLINE RATE
%SOME ARE FAST, SOME ARE SLOW :\
The Iax lack an observable second maxima in the near-infared (NIR),
which would be expected in Type Ia \citep{li03}. 
They have a lower ejecta velocity and do not show hydrogen and helium in their
 spectra.
The appearance of hydrogen and helium only occurs when there are environmental factors, such as  
a circumstellar disks or intermediate mass elements donating the material \citep{fol09}. 
Though the explosion mechanics of Iax are still unknown, the most accurate model
predicts a partial deflagration from the accretion of a Helium star 
onto a C/O WD \citep{krom13, fol1304}.

The host galaxies environments are significant because
the metallicity of the host galaxy affects the luminosity and the age of the
SNe explosion.
Iax have been identified in predominately late-time galaxies, 
but have not been observed in elliptical galaxies \citep{fol1304}. 
This is different from Iax, which are impartial to the galaxy type \citep{van92}. 
We would expect the SNe in more massive galaxies to have a brighter 
luminosity than SNe from smaller galaxies  \citep{fol1305}. 
Metallicity comes into play because progenitors with high 
metallicity results in lower luminosity light curves for their SNe \citep{tim03}. 
However, we know these from studies of Ia SNe, we can only vaguely frame Iax SNe with this information. 
%To truely understand the behaviors of Iax, we study specific parameters. 
 
%INCLUDE: Information about the affects host galaxies have on SNe.
%This will be useful when comparing the ages and seeing if the
%galaxies were important factors (metallicity and luminosity)

Though some parameters and features are known, there is still a lot 
that is not understood about Iax SNe.
The parameter that will be examined in this paper is the ages 
of the surrounding stellar population of these Iax.
By studying the ages of the remaining stars around the SNe,
we can predict the age of the Iax prior to detonation and 
see if there is a correlation of ages.  
%; however, SNe Iax are different from SNe Ia. 
%In the recent decade there has been an increase of SNe Ia analysis. 
%These studies reveal a range of behaviours and spectral features. 

In this analysis we study SN2008ge, SN2008ha, 
SN2010ae, SN2010el, SN2012Z, and SN2014dt. 
The focus was on creating an age limit for the stellar populations around these SNe.
The stellar population around SN2008ha was analysed in fig. 4 of \citet{fol1409}, 
the model recreated in this paper so as to apply the same method 
of analysis to SN2008ge, SN2010ae, SN2010el, SN2012Z, and SN2014dt. 

In this paper, we will discuss our findings of Iax stellar populations.
% (this is a weird sentence).
In section 2, we will discuss the Spectroscopy and Photometry of these SNe .
%and compare them with earlier results. 
We will discuss the methods of our analysis in section 3. 
In section 4, we will discuss our contraints and results.
The summary of this project is in section 5. 
The World Coordinate System (WCS) is used throughout this analysis. 
%Reddening shift of the MilkyWay and host galaxies are used
%to measure the extinction of the spectra.

%(SNe Ia), have been studied in depth, 
%moreso in the last decade due to the increase in technology and 
%an interest in the accelerating universe.
%Type Ia are used to measure the rate of expansion of the universe
%(Perlmutter et al. 1999) because of their near uniform light curves
%and relatively similar spectra. 
%However, not all thermonuclear explosions share this behaviour. 
%Continued study of thermonuclear explosions reveal that they are not a "simple"
% catagory of SNe, they show a wide range of behaviours and spectral features. 
%Continued study of thermonuclear explosions reveal a range of behaviours 
%and spectral features. 


%The behaviors of SNe with peculiar features were cataloged as SNe Iax.
%The unique spectral features and low luminosity lightcurves require a revised class
%thermonuclear explosions. 
%Therefore, SNe Iax are a separate category from SNe Ia.
%The unique features of Iax indicate that they are not a subset of Ia, 
%but should instead be considered their own class. 
%The differences between Ia and Iax indicate that 
%However, Iax are predicted to be a unique "set" of thermonuclear explosions. 
%While the mechanics of this model have not been finalized, 
%the spectra and lightcurves of Iax show that it is a distinct type of SNe
%explosion, and not a subset of Ia. 

%Earlier literature called Iax, 'SN2002cx-like'. SN2002cx was examined
%because of it's [peculiar characteristics.] 
%The current class of Iax include 25 SNe.
%These SNe have a range of behaviors but do have low ejecta velocities, low light 
%curve luminosities, and lack the (usual) Ia features.  



%WHAT OTHER PAPERS HAVE STUDIED THIS - citations

\section{Class Members}
The catalog of SNe is growing and we are able to identify new SNe 
as Iax. Of the list of identified Iax SNe, we will focus on the following:  
SNe 2008ge, 2008ha, 2010ae, and 2010el. %, 2012Z, and 2014dt. 

These SNe are representative of the variety of environments Iax
occur in. 
The host galaxies vary from S0 to spiral galaxies. 
Using these unique environments we will be able to 
see the affects of the environment on these SNe. 
%The increase in new SNe data shows the variations in SNe spectra. 
%The classification of SNe was created when the data regarding Ia and II %
%SN showed unusual variations. 
%SN Iax are a result of the new classifications, Iax had a much lower
% light curve, (exact value), and other stuff ().  
%The current list of SNe Type Iax contain 25 SNe,
%of that list, 6 SNe were chosen to study in depth. 
% was determined to be a class Iax 
% Parameters

%2002cx and 2005hk were blueshifted velocity ,
% 2008A redshifted velocity because it had a unique velocity,
% this implies %that velocity shifts are unique for each SN
%5.  http://adsabs.harvard.edu/abs/2013arXiv1309.4457M

\begin{table}[ht]
\begin{center}
\caption{Supernovae Host Galaxies} 
\begin{tabular}{l*{6}{c}r}
\hline\hline
SN name & R.A. (J2000) & Dec. (J2000) & Host Galaxy  & Morphology & Reference  \\% &?  & ?\\
\hline
2008ge   & 04h 08m 24.68s & -47d 53m 47.4s    & NGC 1527           & SAB0$^{-r}$     & 1 \\%& ? &?\\
2008ha   & 23h 34m 52.7s   & +18d 13m 35s     & UGC 12682         &  Im                    & 1 \\%& ? &?\\
2010ae   & 07h 15m 54.6s   & -57d 20m 37s      & ESO 162- G 017  & Sb? pec edge-on & 1 \\%& ? &?\\
2010el    & 04h 19m 58.8s   & -54d 56m 39s      & NGC 1566           & SAB(s)bc           & 1 \\%& ? &?\\
%2012Z     & 03h 22m 05.3s   & -15d 23m 16s      & NGC 1309           & SA(s)bc             & 1 \\%& ? &?\\
%2014dt    & 12h 21m 57.57s  & +04d 28m 18.5s & M61 or NGC 4303 & SAB(rs)bc         & 1 \\%& ? &?\\
\hline
\end{tabular}
\label{tab:multicol}
\end{center}
\tablenotemark{{\bf References.}(1)}
%\tablenotetext{ALPHA TAG}{TEXT}
\end{table}
%(1) 
%RC3.9
%1991 vol. p. 
%DE VAUCOULEURS, G., DE VAUCOULEURS, A., CORWIN JR., H.G., BUTA, R. J. PATUREL, G., AND FOUQUE, P.
%THIRD REFERENCE CATALOGUE OF BRIGHT GALAXIES, VERSION 3.9 

\subsection{SN 2008ge}

\begin{figure}[htp]
    \centering
	%%%%%%%%%%%%trim = left bottom right top
    \includegraphics[trim = 0mm 150mm 0mm 0mm, clip, scale =0.4]{../Figures/sn08ge_Galaxy.png}
    \caption{False color image of NGC 1527, host galaxy for SN 2008ge.We present a false color image, F814W, F625W, and F435W, red, green, and blue respectively.}%make this more original
    \label{08gegal}
\end{figure}

\begin{figure}[htp]
  \centering

  \subfloat[F435W waveband]{\label{figur:r435}\includegraphics[width=80mm]{../Figures/sn08ge_Regions435.png}}
  \vspace*{\fill}  
  \subfloat[F555W waveband]{\label{figur:r555}\includegraphics[width=80mm]{../Figures/sn08ge_Regions555.png}}
  \\
  \subfloat[F625W waveband]{\label{figur:r625}\includegraphics[width=80mm]{../Figures/sn08ge_Regions625.png}}
  \vspace*{\fill}  
  \subfloat[F814W waveband]{\label{figur:r814}\includegraphics[width=80mm]{../Figures/sn08ge_Regions814.png}}
   \label{fig:Regions08ge}\caption{Surrounding Region of SN 2008ge}
\end{figure}

\begin{figure}[htp]
  \centering

  \subfloat[F435W waveband]{\label{figur:s435}\includegraphics[width=80mm]{../Figures/sn08ge_source435.png}}
  \vspace*{\fill}
  \subfloat[F555W waveband]{\label{figur:s555}\includegraphics[width=80mm]{../Figures/sn08ge_source555.png}}
  \\
  \subfloat[F625W waveband]{\label{figur:s625}\includegraphics[width=80mm]{../Figures/sn08ge_source625.png}}
  \vspace*{\fill}  
  \subfloat[F814W waveband]{\label{figur:s814}\includegraphics[width=80mm]{../Figures/sn08ge_source814.png}}
  \label{fig:Sources08ge}\caption{SN 2008ge Location}
\end{figure}

The first obervations of SN 2008ge were taken by
Chase on 2008 October 8.27 in NGC 1527 \citep{pig08}. 
NGC is a class SAB0$^{-r}$, S0, galaxy, a lenticular galaxy. 
SN 2008ge was observed after its maxium light, the observered peak was 
$M_{V,peak} = -17.4$ mag.% \citep{fol1011}.
Previous observations of NGC 1527 do not show 
signs of star formation and a lack massive stars near 
position of the SNe. 
This along with a "large generated 56Ni mass all suggest a WD progenitor"  \citep{fol1011}.


\subsection{SN 2008ha}

\begin{center}
\begin{figure}
\begin{minipage}[c][5cm][t]{.52\textwidth}
  %\vspace*{\fill}
  %\centering
  \includegraphics[scale = 0.5]{../Figures/sn08ha_Galaxy.png}
  \label{fig:08hagal}
\end{minipage}%
\begin{minipage}[c][5cm][t]{.49\textwidth}
   %\vspace*{\fill}
  %\centering
  \includegraphics[scale = 1.539]{../Figures/sn08ha_Regions2.png}
  \label{fig:r08ha}\par%\vfill
  \includegraphics[scale = 1.539]{../Figures/sn08ha_Source2.png}
  \label{fig:s08ha}
\end{minipage}
\caption{These are false color images of UGC 12682, the host galaxy of SN 2008ha. The colors
red, green and blue correspond to the F814W, F625W, and F435W wavebands respectively. \textit{Left:} This is a }

\end{figure}  
\end{center}
SN 2008ha was first observed 2008 November 7.17 in UGC 12682, 
an Im type highly irregular galaxy, by POSS \citep{puc08}.
The first observations of SN 2008ha classified it as 
a Ia SN with properties simiar to SN 2002cx-like \citep{fol08}. 
At its peak SN 2008ha was observed with a $M_{V,peak} = -14$ mag, 
whereas SN 2002cx was observed at  $M_{V,peak} = -20$ mag. 
This puts 08ha on the extreme end of this classification. 
It is less luminous and has one of the smallest ejecta velocity, ~3000$km/s$ lower than SN 2002cx. 
 \citep{fol09,fol1001,val09}.

%An interesting character. It is an extreme case for IaX because of its low luminosity (-14.21 +/- .15 mag), slower ejecta velocity, KE, and Total Enegy. Maybe had a massive progenitor. This SNe is discussed a lot, how come it wasn't made into it's own class. Compared with 2005E, however 2005E isn't IaX. 
\begin{centering}
\subsection{SN 2010ae}
\begin{figure}
\begin{minipage}[c][5cm][t]{.6\textwidth}
  \vspace*{\fill}
  \centering
  \includegraphics[scale = 0.5]{../Figures/sn10ae_Galaxy.png}
  \caption{test figure one}
  \label{fig:10aegal}
\end{minipage}%
\begin{minipage}[c][5cm][t]{.5\textwidth}
  \vspace*{\fill}
  \centering
  \includegraphics[scale = 1.539]{../Figures/sn10ae_Regions2.png}
  \label{fig:r10ae}\par%\vfill
  \includegraphics[scale = 1.539]{../Figures/sn10ae_Source2.png}
  \label{fig:s10ae}
\end{minipage}
\end{figure}
\end{centering}
SN 2010ae was first observed on 2010 February 23 in ESO 162-G017 \citep{pig10}, a sprial barred galaxy.
SN 2010ae was initially thought to be a Ia SN; however, that classification
was later redacted when analysis of its spectra showed strong similarities to
SN 2008ha, it was then reclassified as SNe Iax \citep{str1003}.

We are observing ESO 162-G017 edge on, we are not sure of the locations of the neighboring stars
relative to the SNe explosion point. 
We will consider if these stars are in the forground or if they lie in the same plane as the SNe.

%but these are all considerations we must make in order to understand our analysis.  
%In this analysis we will be looking at the stellar populations surrounding the SNe point. 

\subsection{SN 2010el}

SN 2010el was first obervered on 2010 June 19 in NGC 1566, 
an intermediate sprial galaxy \citep{mon10}. 
It was classified a type Iax because its
spectra resembled that of SN 2008ha \citep{bes10}.

%\subsection{SN 2012Z}

%SN 2012Z was first observed on 2012 January 29 in NGC 1309,
%a normal spiral galaxy, by LOSS.
%SN 2012Z was classified as Iax when initial observations of its spectra
%displayed properties similar to Iax (Cenko et al. 2012).

%\subsection{SN 2014dt}
%SN 2014dt was first found on 2014 September 29.8 in M61 (Nakano & Itagaki
%(2014) a). 
% It was quickly classified as Type Iax and soon after was  observered with the HST. 
%The HST was able to collect data soon after the initial discovery. ()

%9. http://adsabs.harvard.edu/abs/2014A%26A...561A.146S  
%Hα and [Nii] λλ6548, 6583 emission, 12 + log (O/H) = 8.40 ± 0.18 dex
%O3N2 method suggests an averaged local metallicity of 12 + log(O/H) = 8.34±0.14 dex, where again the quoted uncertainty includes measurement and systematic errors. These oxygen abundance metallicities correspond to 0.52 and 0.44 the known solar metallicity of ∼8.69 dex (Asplund et al. 2009), and so are consistent with the metallicity of the LMC.

\section{Spectroscopy \& Photometry}

This anaysis used photometric images from the Hubble Space Telescope (HST).
%The HST collected these images using 
%(HST Info- exposures, camera lens size 
%and date and time it observed each object.
%I should include images of the SN discovery- tho not relevant)
The SNe were observered in the F435W, F555W, F625W, and F814W wavebands. 
The cosmic rays were removed using the L.A. Cosmic Algorithm,
the resulting images were used in the remainder of the analysis. 

%SNe 2008ge, 2008ha, 
%2010ae, 2010el, 2012Z, and 2014dt. 
% IRAF DAOPHOT p

%This data was collected [DATE] using [type of collection method, ccd stuff,
%how the images were corrected]. 

%IRAF (Image Reduction and Analysis Facility) is distributed by the National
%Optical Astronomy Observatory, which is operated by the Association
%of Universities for Research in Astronomy, Inc., under cooperative agreement
%with the National Science Foundation.

\section{Methods}
Catalogs of the surrounding stellar population 
were generated for each SNe using DolPhot on HST photometric images. 
The objects in the catalog were then filtered by their 
S/N, sharp, crowd, and round parameters to identify real sources.
%The filters included bright and faint sources over specific S/N thresholds, 
%depending on the environment of the host galaxy. 
The parameters were adjusted for each SNe environment to extract real stars sources from the catalogs.

We examined the environment around SN 2008ha, its 
Color Magnitude Diagram (CMD) had been made in Fig 4. of \citet{fol1409}.
This paper modeled concentric regions of 15, 30, and 45 pix (75, 150, 225 pc)
around the SNe. 
This SN was only reddened by the Milky way, there was no host galaxy reddening.
For SN 2008ha we used $ 12 + Log(O/H) = 8.16, Z = 0.006$ \citep{fol09,tak95}.
With our catalog of stars, we were able to recover the same
results as \citet{fol1409}.
Fig () shows the same sources from different wavebands being plotted
and a similar age limit. 
The ages of the stellar population in our paper are in agrement with the earlier paper. 
%The remaining criteria included it's position, the filters included sources within a radius 
%100 pixels, though the physical distance is dependent on the SNe being discussed. 
%The other criteria was the Signal to Noise (S/N), at minimum, 
%sources had to be above S/N of 3 in at least one waveband. 
%The sources also needed to have particular Crowding values, for example, 
%clustered sources may included extended sources, which would not be "good" 
%sources. The sharpness and roundness was also taken into account. 
%The sharpness was set to [], and the roundness was set to []. 
%Using these parameters, we were able to recreate the Color Magnitude Diagram (CMD)
%of SN 2008ha, figure () from (Foley ...). This study focused on the stars within
% (15, 30, and 45 pix or 730, 1500, 2200 AU)

SN 2008ge was observed 6.55 arcsec away from the center of it's host galaxy,
NGC 1527, a bright lenticular galaxy. 
In order to look at the region of space around the coordinates of the SN, 
we made a subracted image for each of the wavebands, this was essential in
properly identifying star-like sources. 
The subtracted images
were created by splitting the galaxy into four quadrants, and subtracting 
cross quadrants. 
The subtracted images are shown in Figures ~\ref{fig:Regions08ge} \& ~\ref{fig:Sources08ge}, 
highlighted are the real sources from the catalog. 
We estimated the metallicity for SN 2008ge to be between 
$Z = .02$ dex and $Z = .03$ dex, $12 + Log(O/H) = 8.68$  and $12 + Log(O/H) = 8.85$ respectively \citep{pro11}. 
There was no host galaxy reddening for this SN, though we included Milky Way reddening in our analysis. 

Other than SN 2008ge, we were able to use the lacosmic.fits files in the analysis of SN 2010ae and SN 2010el. 
Using DolPhot we were able to extract sources from the HST images. 
Using the new catalogs for SNe 2010ae and 2010el,
real sources were identified, either through contraints on crowding, sharpness, and roundness,
or through a visual examination. The results from this analysis will be discussed in the next section. 
We included both host galaxy and milky way reddening in our final analysis. 

SN 2010ae was obsevered in an edge on galaxy,
we had to be very careful with the stars we used because it would be hard
to tell if something was in the plane of the galaxy or if it was somewhere in the foreground. 
SN 2010ae had a metallicity of $ 12 + Log(O/H) = 8.36, Z = 0.0096$. It was reddened by its host galaxy
as well as by the milky way. 

SN 2010el was highly reddened by it's host galaxy, we included the reddening vector to demonstrate
the affects of the host galaxy on the stellar population. 
This region had an oxygen abundance of $ 12 + Log(O/H) = 8.73, Z = 0.0224.$

%There are different host galaxies environments and thus, different metallicity values. 


%The NGC 1527 is bright and identifying sources in the HST images was difficult
%The catalog was made of S/N fluctuations, which aren't as helpful as you would think.
%So the catalog was contrained as much as I could reasonably make it. 
%This meant looking at bright and dim stars (though saying they're stars is really
%pushing the definition because sometimes it was just a flucuation). 
%Following that was excluding objects that were within 25pix of the center
%of the galaxy. This space was not identified well in the catalog. 
%Using the subtracted images of the wavebands, we are able to see 
%details of what is happening near center of the galaxy. 
%After contraining parameters, such as S/N, crowd, sharpness, and roundness, 
%we also visually searched for sources from the remaining list. 
%Yes, super tiresome and really tedious. Oh well. 
%These steps took a region of space with over 1000 sources
%and reduced it to less than 30. 


\section{Discussion \& Results}


SNe Iax are a new category of SNe, they include a wide range of SNe with unique parameters.
%Age was the parameter studied in this paper. 
The age of the host galaxies can be a factor which separates 
SNe Ia from Iax.
By creating CMDs for the neighboring stellar populations,
we are able to measure the ages of the stars in that region. 
This allowed us to estimate the age the SN would have been prior to its detonation. 


The ages of SNe 2008ge, 2008ha, 2010ae, and 2010el were found to be
%, 2012Z, and 2014dt 
19.5-27.5 Myr, 52.5 Myr, 26.3 Myr, 65.0 Myr respectively. 
From these results, there is spread of ages between Iax. 
This information indicates that Iax come from young stars, though
%this tells us is that there is a relation between these SNIax,
our current set of data only includes 4 values.
this gives a weak relationship which would be strengthened by recreating this analysis for the remaining SN Iax.
Further cataloging of Iax will add to the information we have and give us a better model for these types of SNe.  


{\it Acknowledgments:} We are grateful to **ACKNOWLEDGEMENTS HERE***.

%\bibliographystyle{apalike}
%\bibliography{SNeIaxCite}{}

\bibliography{SNeIaxCite}{}
\bibliographystyle{apalike}%don't know why this works now
%\bibliographystyle{apj}
%\bibliographystyle{plainnat}% this one works
%\bibliographystyle{apacite}
\end{document}